\ifthenelse{\equal{\MajorFormat}{cs}}
{
    \chapternonum{毕业论文(设计)文献综述和开题报告考核}
    \bfseries

    {
        \zihao{4}
        \noindent 导师对开题报告、外文翻译和文献综述的评语及成绩评定:
    }

    \vspace{50mm}

    % Example usage:
    % \thesisproposaleval[10][15][5]
    \thesisproposaleval[][][]

    \signature{导师签名}

    {
        \zihao{4}
        \noindent 学院盲审专家对开题报告、外文翻译和文献综述的评语及成绩评定:
    }


    \mbox{} \vfill
    \thesisproposaleval
    \signature{开题报告审核负责人(签名/签章)}
}
{
    % =============================================================
    % 去掉引导线
    % \addtocontents{toc}{\cftpagenumbersoff{chapter}} 
    % \phantomsection
    % \addcontentsline{toc}{chapter}{《浙江大学本科生文献综述和开题报告考核表》} 
    % \addtocontents{toc}{\cftpagenumberson{chapter}}
    % =============================================================

    % =============================================================
    % 保留引导线
    \addtocontents{toc}{\protect\renewcommand{\protect\cftchappagefont}{\protect\color{white}}}
    \phantomsection
    \addcontentsline{toc}{chapter}{《浙江大学本科生文献综述和开题报告考核表》}
    \addtocontents{toc}{\protect\renewcommand{\protect\cftchappagefont}{}}
    % =============================================================

    \begingroup
        \renewcommand{\addcontentsline}[3]{}             
        \sectionnonum[openany]{毕业论文(设计)文献综述和开题报告考核}
    \endgroup

    \bfseries

    {
        \zihao{4}
        \noindent 一、对文献综述、外文翻译和开题报告评语及成绩评定:
    }


    \mbox{} \vfill
    \thesisproposaleval
    \signature{开题报告答辩小组负责人(签名)}
}
