% =================================================================
% 去掉引导线
% \addtocontents{toc}{\cftpagenumbersoff{chapter}} 
% \phantomsection
% \addcontentsline{toc}{chapter}{《浙江大学本科生毕业论文(设计)任务书》} 
% \addtocontents{toc}{\cftpagenumberson{chapter}}
% =================================================================

% =================================================================
% 保留引导线
\addtocontents{toc}{\protect\renewcommand{\protect\cftchappagefont}{\protect\color{white}}}
\phantomsection
\addcontentsline{toc}{chapter}{《浙江大学本科生毕业论文(设计)任务书》} 
\addtocontents{toc}{\protect\renewcommand{\protect\cftchappagefont}{}}
% =================================================================

\begingroup
    \renewcommand{\addcontentsline}[3]{} 
    \sectionnonum[openany]{\zihao{2}\huawenxingkai 本科生毕业论文(设计)任务书}
\endgroup

{
    \bfseries
    \huawenfangsong
    \zihao{-4}
    \noindent 一、题目:\\
    \noindent 二、指导教师对毕业论文(设计)的进度安排及任务要求:\\

    \vskip 50mm

    \noindent 起讫日期 20 \quad 年 \quad  月 \quad  日 \quad 至 \quad 20 \quad  年 \quad  月  \quad 日
    \begin{flushright}
        \bfseries \zihao{-4}
            指导教师(签名) \underline{\multido{}{5}{\quad}} 职称 \underline{\multido{}{5}{\quad}}
    \end{flushright}

    \noindent 三、系或研究所审核意见:\\

    \mbox{} \vfill
    \signature{负责人(签名)}
}
