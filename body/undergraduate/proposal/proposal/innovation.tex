\newpage

\section{主要创新点}
基于“预测与优化一体化”的前沿科学范式,本研究针对SPO框架的预测准确性缺失与环境鲁棒性不足,进一步完善其性能表现,主要创新点可概括如下:

创新点1:提出了

提出了基于混合损失函数的稳定化预测优化方法,解决了端到端模型参数物理意义缺失的难题。 现有SPO框架虽然通过引入决策损失(SPO+ Loss)解决了传统“预测然后优化”范式中的目标失配问题,但其单纯追求决策后悔值最小化的训练机制往往导致预测参数发生严重的数值漂移,丧失了作为物理参数(如成本、时间)的真实含义与可解释性 。本研究创新性地构建了一种融合统计预测误差(MSE)与决策后悔值的混合损失函数,将预测准确性作为正则化约束显式纳入优化目标。该方法在保留利用优化问题梯度信息进行方向性修正的同时,强制模型参数回归真实的物理分布。这不仅有效克服了端到端学习中的“多重最优性”陷阱,恢复了模型的物理可解释性,还显著提升了模型在处理多重共线性特征时的数值稳定性。

2. 构建了面向非平稳环境的分布鲁棒SPO决策框架,突破了传统模型仅适应独立同分布数据的局限。 针对现有预测驱动最优化方法大多基于“风险中性”与“独立同分布(i.i.d.)”假设,难以应对现实中训练集与测试集分布不一致(Distribution Shift)的问题 ,本研究创新性地将分布鲁棒优化(DRO)理论引入端到端学习框架。不同于传统经验风险最小化(ERM)策略,本研究利用条件风险价值(CVaR)或Wasserstein距离构建包含真实分布的模糊集,推导了面向“最坏情况分布”的鲁棒SPO损失函数的凸松弛形式。该框架赋予了模型在训练阶段对抗环境扰动的能力,使其能够有效抵御数据分布偏移带来的性能衰减,填补了SPO理论在非平稳环境下鲁棒性研究的空白。

3. 建立了一套兼顾预测保真度与环境适应性的通用化端到端决策范式,并通过典型组合优化场景验证了其有效性。 本研究打破了以往研究中“预测精度”与“决策质量”相互割裂甚至对立的僵局,通过整合混合损失机制与分布鲁棒优化,提出了一套“既准又稳”的统一决策范式。以经典的随机最短路问题为实验载体,本研究设计了包含同分布与分布偏移的多维度仿真实验,系统揭示了该范式在不同噪声水平与偏移强度下的性能边界。这一工作不仅在理论上完善了数据驱动决策的方法论体系,更为物流调度、电网管理等高风险、高可靠性要求的实际管理问题提供了可落地的算法支撑与解决方案 。